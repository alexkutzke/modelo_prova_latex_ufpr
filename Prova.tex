\documentclass[a4paper, 12pt, addpoints]{cls/exam}
%\documentclass[a4paper, 11pt, addpoints, answers]{exam}  % Desconmente esta linha, para ver as respostas, e comente a de cima
\usepackage{sty/UFPR}
\usepackage{listings} % Mostrar código-fonte
\usepackage[brazil]{babel}
\usepackage{multicol}
\usepackage{booktabs}
\usepackage{here}

\setlength{\columnsep}{25pt}
\lstdefinestyle{js}{
    basicstyle=\ttfamily,
    breaklines=true,
    breakatwhitespace=true,
    tabsize=1,
    resetmargins=true,
    xleftmargin=0pt,
    frame=none
}

%%%%%%%%%%%%%%%%%%%%%%%%%%%%%%%%%%%%%%%%%%%%%%%%%%%%

\pointpoints{Ponto}{Pontos}

\begin{document}

\nomeProfessor{Alexander Robert Kutzke}
\nomeCurso{Análise e Desenvolvimento de Sistemas}
\nomeDisciplina{}
\semestre{}   % Deixe em branco se for para mais de um semestre (recuperação)
\dataDaProva{ \today }
\tipoAvaliacao{Primeira avaliação}

\info
%\vspace{-1.5 cm}

\begin{small}
\noindent Orientações gerais:
\begin{itemize}
  \itemsep=-2mm
  \item A interpretação faz parte da prova;
  \item Seja claro nas explicações;
  \item A prova deve ser feita à caneta; 
\end{itemize}
\end{small}
\hrulefill

%%%%%%%%%%%%%%%%%%%%%%%%%%%%%%%%%%%%%%%%%%%%%%%%%

%%AAG: tipos de questões:
%               - \begin{checkboxes}
%               - \begin{choices}
%               - \begin{oneparchoices}
%               - \begin{oneparcheckboxes}
% vide exam.cls 3779

\begin{questions}

% =============================================================================
% \question[2] Explique, linha a linha, o código PHP abaixo:

% \lstset{
% 	numbers=left,
% 	language=PHP,
% 	frame=single,
% 	breaklines=true,
% 	basicstyle=\footnotesize
% }

% \begin{lstlisting}
%  <?php
%  session_start();

%  if ( !isset($_SESSION['login']) and isset($_SESSION['senha']) ){
%    session_destroy();
%    unset($_SESSION['login']);
%    unset($_SESSION['senha']);
%    header('location:login.php');
%  }
%  ?>
% \end{lstlisting}


\end{questions}
\end{document}
