\documentclass[a4paper, 12pt, addpoints]{cls/exam}
%\documentclass[a4paper, 11pt, addpoints, answers]{exam}  % Desconmente esta linha, para ver as respostas, e comente a de cima
\usepackage{sty/UFPR}
\usepackage{listings} % Mostrar código-fonte
\usepackage[brazil]{babel}
\usepackage{multicol}
\usepackage{booktabs}
\usepackage{here}

\setlength{\columnsep}{25pt}
\lstdefinestyle{js}{
    basicstyle=\ttfamily,
    breaklines=true,
    breakatwhitespace=true,
    tabsize=1,
    resetmargins=true,
    xleftmargin=0pt,
    frame=none
}

%%%%%%%%%%%%%%%%%%%%%%%%%%%%%%%%%%%%%%%%%%%%%%%%%%%%

\pointpoints{Ponto}{Pontos}

\begin{document}

\nomeProfessor{Alexander Robert Kutzke}
\nomeCurso{Análise e Desenvolvimento de Sistemas}
\nomeDisciplina{TI 161 - Desenvolvimento de Aplicações WEB 1}
\semestre{}   % Deixe em branco se for para mais de um semestre (recuperação)
\dataDaProva{\today}
\tipoAvaliacao{Primeira avaliação}

\info
%\vspace{-1.5 cm}

%%%%%%%%%%%%%%%%%%%%%%%%%%%%%%%%%%%%%%%%%%%%%%%%%

%%AAG: tipos de questões:
%               - \begin{checkboxes}
%               - \begin{choices}
%               - \begin{oneparchoices}
%               - \begin{oneparcheckboxes}
% vide exam.cls 3779

\begin{questions}

% =============================================================================
\question[1] Explique, de maneira resumida, como funciona o paradigma de programação para a WEB (paradigma cliente-servidor). Indique qual o objetivo do protocolo HTTP (HyperText Transfer Protocol) e suas relações com o desenvolvimento WEB.

% =============================================================================
\question[1] Explique o que é o HTML (HyperText Markup Language) e suas principais características.

% =============================================================================
\question[1] Explique o que é o CSS (Cascading Style Sheets) e qual a sua função no desenvolvimento WEB.

% =============================================================================
\question[1] Explique o que é o Javascript e qual a sua função no desenvolvimento WEB.

% =============================================================================
\question[1] Explique o que é o PHP e como ele pode ser utilizado no desenvolvimento WEB.

% =============================================================================
\question[2] Explique o que são e como funcionam as sessões (Sessions) do PHP. Qual a relação desse recurso com os “Cookies” do protocolo HTTP?

% =============================================================================
\question[3] Considere que você é um desenvolvedor de uma aplicação Web em que é necessário atualizar um cadastro de um cliente (nome, email, profissão, cpf, etc.). Uma tela dessa aplicação exibe a lista de todos os clientes cadastrados. Nesta página, cada nome de cliente possui ao lado link chamado “alterar”, o qual direciona para a página que deve conter o formulário com os dados atuais do cliente para que o usuário possa alterá-los. Explique, em linhas gerais, como você implementaria esse processo de alteração dos dados de um cliente, a partir do clique no link “alterar” (considere que cada usuário possui uma identificação numérica, o id). Além disso, Identifique em quais locais você utilizaria os métodos GET ou POST do protocolo HTTP. A tabela abaixo simula o conteúdo da página de listagem dos clientes:

\begin{table}[H]
\centering
\caption{Exemplo de listagem dos clientes.}
\label{my-label}
\begin{tabular}{@{}lll@{}}
\toprule
Id & Cliente       & Opções  \\ \midrule
1  & Homer Simpson & alterar \\
2  & Marge Simpson & alterar \\
3  & Bart Simpson  & alterar \\ \bottomrule
\end{tabular}
\end{table}

% =============================================================================
\question[2] Explique, linha a linha, o código PHP abaixo:

\lstset{
	numbers=left,
	language=PHP,
	frame=single,
	breaklines=true,
	basicstyle=\footnotesize
}

\begin{lstlisting}
 <?php
 session_start();

 if ( !isset($_SESSION['login']) and isset($_SESSION['senha']) ){
   session_destroy();
   unset($_SESSION['login']);
   unset($_SESSION['senha']);
   header('location:login.php');
 }
 ?>
\end{lstlisting}

% =============================================================================
\question[2] Considere o seguinte código HTML: 

\lstset{
	language=HTML
}


\begin{lstlisting}
<!DOCTYPE html>
<html>
<head>
<style>
.container {
    position: relative;
}


.classex {
    position: absolute;
    bottom: 8px;
    right: 16px;
    font-size: 18px;
}


img { 
    width: 100%;
    height: auto;
}
</style>
</head>
<body>


<h1>Prova Web 1</h1>


<div class="container">
  <img src="imagem.jpg" width="1000" height="300">
  <div class="classex">Algum texto qualquer</div>
</div>


</body>
</html>
\end{lstlisting}

Faça um esboço de como essa página seria renderizada pelo Navegador. Indique os diferentes elementos e seus respectivos posicionamentos.

\end{questions}
\end{document}
